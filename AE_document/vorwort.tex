\chapter{Vorwort}
\label{ch:1}

\section{Aufgabenstellung und Struktur des Dokuments}
\label{sec:1.1}

\subsection*{Aufgabenstellung}

{Im Rahmen des Softwareentwicklungspraktikums (CES$\_$SS2012) soll eine Software zur Simulation eines Stehaufkreisels erstellt werden.
Die Simulation soll dabei einmal f\"ur den Fall verschwindender
Reibung sowie einmal für den Fall einer reibungsbehafteten Bewegung durchgef\"uhrt werden. 

Als Programmiersprache soll C++ verwendet werden. Der Quellcode soll derart strukturiert
und kommentiert sein, dass sp\"atere Modifikationen und Erweiterungen durch weitere
Mitarbeiter m\"oglich sind.}

\section{Projektmanagement}
\label{sec:1.2}

{\em 
\begin{tabular}{l l}
	Protoyping (MATLAB/ FORTRAN)& Alexander \\
	Dokumentation &Lena \\
	C++ Code:& \\
	1. Solver+RHS& \\
	2. Schnittstellenspezifikation & \\
	3. GUI & \\
	4. Exception-Handling & \\
\end{tabular}
}

\section{Lob und Kritik}
\label{sec:1.3}

{\em Der Niko ist voll cool}

