\chapter{Analyse}
\label{ch:2}

\section{Anforderungsanalyse}
\label{sec:2.1}

\subsection{Benutzeranforderungen}

{Das von Herrn Professor Gauger gestellte Simulationsproblem umfasst die Erstellung einer Software ... .
Im Speziellen wird ein Runge-Kutta56-Verfahren mit adaptiver Schrittweitensteuerung unter Betrachtung einer Erhaltungsgröße (conserved quantity) zur Simulation des Problems verwendet. Die Realisierung der Simulation findet in C++ statt.

Die Bedienung sowie die Auswertung der Ergebnisse der Simulationssoftware muss durch eine grafische Benutzeroberfläche (GUI) möglich sein.

Die Simulationsergebnisse können in einer ASCII formatierten Datei zur weiteren Verarbeitung und Auswertung exportiert werden.

(Bilder exportiere?) 

Durch den modularen Aufbau ist die Wartbarkeit und spätere Modifikationen oder Erweiterungen durch externe Mitarbeiter gewährleistet. 


Im weiteren umfasst der Arbeitsauftrag die Erstellung einer Benutzerdokumentation für den Endanwender. 
\em Rechte-Seite

Hauptproblem besteht im Lösen der Rechten Seite (mit rkv56) }

\subsection{Anwendungsfallanalyse}

{\em Anwendungsf\"alle (Statik: Anwendungsfalldiagramme; Dynamik: 
Aktivit\"atsdiagramme; Textuelle Beschreibungen laut Vorlage}

\subsubsection{Systemanforderungen}

{\em Beschreibung funktionaler und nichtfunktionaler Anforderungen basierend
auf Anwendungsf\"allen (Ziel: Konsens zwischen Auftraggeber und Auftragnehmer)}

\subsubsection{Benutzerdokumentation}

{\em kann basierend auf den Anwendungsf\"allen erstellt werden; siehe
\refchapter{ch:4}}

\section{Begriffsanalyse}

{\em Identifikation von Klassenkandidaten basierend auf Anforderungen;
Assoziationen zwischen Klassenkandidaten (Aggregation, Komposition, Vererbung)
und Kardinalit\"aten}


