\chapter{Analyse}
\label{ch:2}

\section{Anforderungsanalyse}
\label{sec:2.1}

\subsection{Benutzeranforderungen}

{Das von Herrn Professor Gauger gestellte Simulationsproblem umfasst die Erstellung einer Software zur Simulation eines Stehaufkreisels.

Die Simulation muss sowohl den reibungsbehafteten, als auch reibungsfreien Fall korrekt simulieren.

Im Speziellen wird ein Runge-Kutta56-Verfahren mit adaptiver Schrittweitensteuerung unter Betrachtung einer Erhaltungsgr\"o\ss e (\textit{conserved quantity}) zur Simulation des Problems verwendet.\\ 

Das Rkv56 Verfahren wurde durch ein StepperDopr853-Verfahren ersetzt, um eine h\"ohere Genauigkeit zu erreichen.\\ 
Die Realisierung der Simulation findet in C++ statt. \\

Die Bedienung sowie das Ausgeben der Simulationsergebnisse muss durch eine grafische Benutzeroberfl\"ache (\textit{GUI}) m\"oglich sein.

Die Simulationsergebnisse k\"onnen in einer \textit{ASCII}-formatierten Datei zur weiteren Verarbeitung und Auswertung exportiert werden.

Durch den modularen Aufbau ist die Wartbarkeit und einfache Erweiterbarkeit der Software durch Dritte gew\"ahrleistet.
 

Das Kernproblem besteht im L\"osen der Rechten Seite des folgenden Differentialgleichungssystems:

\begin{align*}
	&
	\ddot \theta (I+ma^2\sin^2\theta+kma\sin \theta(R-a\cos\theta)(-\dot x_c
	\sin\phi+\dot y_c \cos \phi+(R-a\cos\theta)\dot \theta))
	\\&
	=\underbrace{-(I_3-I)\dot \phi^2\sin\theta\cos\theta}_{=0}-I_3\dot \phi \sin \theta \dot \psi 
	+ (g+a\dot \theta^2\cos \theta)(-ma\sin\theta - km(R-a\cos \theta)
	\\&
	(-\dot x_c \sin \phi +\dot y_c \cos \phi+(R-a\cos\theta)\dot \theta))
\end{align*}
\begin{align*}
	&
	\ddot \phi I \sin\theta= -\underbrace{(2I-I_3)}_{=I}
	\dot \phi \dot \theta \cos \theta + I_3\dot \theta \dot \psi
	\\&
	-km(g+a\cos\theta\dot\theta^2+a\sin\theta\ddot\theta)
	(a-R\cos\theta)(\dot x_c\cos \phi +\dot y_c \sin\phi+
	(a\dot \phi+\dot \psi R)\sin \theta)
\end{align*}
\begin{align*}
	&\ddot \psi I_3=-I_3(\ddot \phi \cos \theta - \dot \phi \dot \theta \sin \theta)
	\\&
	-km(g+a\cos\theta\dot\theta^2+a \sin\theta\ddot \theta)(R\sin\theta)
	(\dot x_c\cos \phi +\dot y_c \sin \phi+(a\dot \phi +\dot \psi R)\sin\theta)
\end{align*}
\begin{align*}
	&m\ddot x_c=-km(g+a\cos\theta\dot \theta^2+a\sin \theta\ddot\theta)
	(\dot x_c+(a\dot \phi + \dot \psi R)\sin \theta\cos \phi+
	(a\cos \theta -R)\sin \phi \dot \theta)
\end{align*}
\begin{align*}
	&m\ddot y_c=-km(g+a\cos \theta\dot \theta^2+a \sin \theta \ddot \theta)
	(\dot y_c+(a\dot \phi + \dot \psi R)\sin \theta \sin \phi
	+(R-a\cos \theta)\cos\phi \dot \theta)
\end{align*}


\subsection{Anwendungsfallanalyse}
\subsection*{Beschreibung der Anwendungsf\"alle}
\begin{tabular}{|p{5cm}|p{2cm}|p{3cm}|}
\hline \\
\textbf{Name} & \multicolumn{2}{p{5cm}|}{Export/Import as Tecplot file} \\
\hline \\
\textbf{Ziel} & \multicolumn{2}{p{5cm}|}{Enable storage of simulation data} \\
\hline \\
\textbf{Einordnung} & \multicolumn{2}{p{5cm}|}{} \\
\hline \\
\textbf{Vorbedingung} & \multicolumn{2}{p{5cm}|}{Simulation has been run/There are existing save files} \\
\hline \\
\textbf{Nachbedingung} & \multicolumn{2}{p{5cm}|}{A save file is created/Simulation data is loaded and graphed} \\
\hline \\
\textbf{Nachbedingung im Fehlerfall} & \multicolumn{2}{p{5cm}|}{Errormessage is shown}\\
\hline \\
\textbf{Haupt-Neben-akteure} & \multicolumn{2}{p{5cm}|}{User} \\
\hline \\
\textbf{Ausl\"oser} & \multicolumn{2}{p{5cm}|}{User presses the \textit{Export Data} or \textit{Import data} button} \\
\hline \\
\textbf{Standardfluss} & \textbf{Schritt}& \textbf{Aktion} \\
\hline \\
& 1 & User presses the \textit{Export Data} or \textit{Import data} button \\
\hline \\
& 2 & User selects a file to export/import \\
\hline \\
& 3 & Export file is created/Simulation data is loaded and graphed \\
\hline
\end{tabular}

\begin{tabular}{|p{5cm}|p{2cm}|p{3cm}|}
\hline \\
\textbf{Name} & \multicolumn{2}{p{5cm}|}{Input Parameters} \\
\hline \\
\textbf{Ziel} & \multicolumn{2}{p{5cm}|}{A set of parameters have been entered} \\
\hline \\
\textbf{Einordnung} & \multicolumn{2}{p{5cm}|}{} \\
\hline \\
\textbf{Vorbedingung} & \multicolumn{2}{p{5cm}|}{} \\
\hline \\
\textbf{Nachbedingung} & \multicolumn{2}{p{5cm}|}{The user can start a simulation} \\
\hline \\
\textbf{Nachbedingung im Fehlerfall} & \multicolumn{2}{p{5cm}|}{Errormessage is shown}\\
\hline \\
\textbf{Haupt-Neben-akteure} & \multicolumn{2}{p{5cm}|}{User} \\
\hline \\
\textbf{Ausl\"oser} & \multicolumn{2}{p{5cm}|}{User presses the \textit{Change Parameter} button in the GUI} \\
\hline \\
\textbf{Standardfluss} & \textbf{Schritt}& \textbf{Aktion} \\
\hline \\
& 1 & User presses the \textit{Change Parameter} button in the GUI \\
\hline \\
& 2 & User enters parameters in a popupwindow \\
\hline \\
& 3 & User clicks \textit{Submit Changes}\\
\hline \\
\textbf{Nebenfluss} & \textbf{Schritt} & \textbf{Aktion} \\
& 2a & User clicks on the Import/Export buttons\\
\hline
\end{tabular}

\begin{tabular}{|p{5cm}|p{2cm}|p{3cm}|}
\hline \\
\textbf{Name} & \multicolumn{2}{p{5cm}|}{Save/Load Parameters from file} \\
\hline \\
\textbf{Ziel} & \multicolumn{2}{p{5cm}|}{Parameters are saved/loaded from file} \\
\hline \\
\textbf{Einordnung} & \multicolumn{2}{p{5cm}|}{} \\
\hline \\
\textbf{Vorbedingung} & \multicolumn{2}{p{5cm}|}{There are parameters to save/there is a file to load} \\
\hline \\
\textbf{Nachbedingung} & \multicolumn{2}{p{5cm}|}{A parameter file is created/parameters are loaded} \\
\hline \\
\textbf{Nachbedingung im Fehlerfall} & \multicolumn{2}{p{5cm}|}{Errormessage is shown}\\
\hline \\
\textbf{Haupt-Neben-akteure} & \multicolumn{2}{p{5cm}|}{User} \\
\hline \\
\textbf{Ausl\"oser} & \multicolumn{2}{p{5cm}|}{User presses the \textit{Save} or \textit{Load} button in the GUI} \\
\hline \\
\textbf{Standardfluss} & \textbf{Schritt}& \textbf{Aktion} \\
\hline \\
& 1 & User presses the \textit{Change Parameter} button\\
\hline \\
& 2 & User presses the \textit{Save} or \textit{Load} button in the GUI \\
\hline \\
& 3 & User clicks \textit{User chooses a file to save/load}\\
\hline \\
\textbf{Nebenfluss} & \textbf{Schritt} & \textbf{Aktion} \\
& 1a & User enters parameters\\
\hline
\end{tabular}

\begin{tabular}{|p{5cm}|p{2cm}|p{3cm}|}
\hline \\
\textbf{Name} & \multicolumn{2}{p{5cm}|}{Start Simulation} \\
\hline \\
\textbf{Ziel} & \multicolumn{2}{p{5cm}|}{Run the mathematical solver with the given parameters} \\
\hline \\
\textbf{Einordnung} & \multicolumn{2}{p{5cm}|}{} \\
\hline \\
\textbf{Vorbedingung} & \multicolumn{2}{p{5cm}|}{Parameters have been entered and checked for validity} \\
\hline \\
\textbf{Nachbedingung} & \multicolumn{2}{p{5cm}|}{Solver is finished, output is being created} \\
\hline \\
\textbf{Nachbedingung im Fehlerfall} & \multicolumn{2}{p{5cm}|}{Solver could not finish calculation}\\
\hline \\
\textbf{Haupt-Neben-akteure} & \multicolumn{2}{p{5cm}|}{User, Solver} \\
\hline \\
\textbf{Ausl\"oser} & \multicolumn{2}{p{5cm}|}{User presses the \textit{Simulate} button in the GUI} \\
\hline \\
\textbf{Standardfluss} & \textbf{Schritt}& \textbf{Aktion} \\
\hline \\
& 1 & User starts the simulation\\
\hline \\
& 2 & Check validity of parameters \\
\hline \\
& 3 & Run the solver\\
\hline
\end{tabular}

\begin{tabular}{|p{5cm}|p{2cm}|p{3cm}|}
\hline \\
\textbf{Name} & \multicolumn{2}{p{5cm}|}{Toggle View} \\
\hline \\
\textbf{Ziel} & \multicolumn{2}{p{5cm}|}{Enable the user to view a different set of graphs} \\
\hline \\
\textbf{Einordnung} & \multicolumn{2}{p{5cm}|}{} \\
\hline \\
\textbf{Vorbedingung} & \multicolumn{2}{p{5cm}|}{There is simulation data to be graphed} \\
\hline \\
\textbf{Nachbedingung} & \multicolumn{2}{p{5cm}|}{} \\
\hline \\
\textbf{Nachbedingung im Fehlerfall} & \multicolumn{2}{p{5cm}|}{Errormessage is shown}\\
\hline \\
\textbf{Haupt-Neben-akteure} & \multicolumn{2}{p{5cm}|}{User} \\
\hline \\
\textbf{Ausl\"oser} & \multicolumn{2}{p{5cm}|}{User presses one of the buttons \textit{Angles},\textit{Derivatives} or \textit{Positions} in the GUI} \\
\hline \\
\textbf{Standardfluss} & \textbf{Schritt}& \textbf{Aktion} \\
\hline \\
& 1 & User presses one of the buttons \textit{Angles},\textit{Derivatives} or \textit{Positions} in the GUI\\
\hline \\
& 2 & A new set of data is graphed according to the button pressed \\
\hline
\end{tabular}

\subsection*{Systemanforderungen}
\subsubsection*{Funktionale Anforderungen}

{Dem Anwender ist es m\"oglich die Simulationsparameter $k$ (Reibung) sowie 
$ \dot \psi (rad/s), \theta (rad), R (cm), a (cm), m (g)$ und die Toleranz der Erhaltungsgr\"o\ss e  \"uber eine grafische Eingabemaske festzulegen. %%% HIER!!!
Wenn w\"ahrend der Simulation ein Fehler auftritt wird der Anwender 
\"uber ein Popup-Fenster benachrichtigt. Nach Durchlauf der Simulation
bekommt der Anwender die Simulationsergebnisse 
\textit{ - $\theta, \psi, \phi, x_c, y_c, \dot \theta, \dot \psi, 
\dot \phi,v_x,v_y$ -} in Form von \textit{LineCharts} in eine \textit{GUI} 
eingebettet angezeigt.} 

Die auf der \textit{GUI} ausgegebenen Plots k\"onnen als Bilddatei oder im Tecplotformat exportiert werden. \medskip

Kommt es w\"ahrend der Laufzeit zu einem kritischen Fehler (ein Fehler, der das korrekte Fortf\"uhren des Programmes unm\"oglich macht) wird der Anwender \"uber ein Popup-Fenster benachrichtigt und das an die Stelle zur\"uckgesetzt, an der der Fehler auftrat.
\medskip

\subsubsection*{Nicht-Funktionale Anforderungen}
{
Die Exportfunktion der Simulationssoftware schreibt Tecplot konforme ASCII-kodierte Ausgabedateien.
Vormals exportierte Dateien k\"onnen wieder importiert und geplottet werden.
Ebenso k\"onnen \"altere Parameterkonfigurationen importiert werden.
}

\section{Begriffsanalyse}
\begin{itemize}
\item LineChart - \textit{Zwei Achsen Diagramm mit Kartesischem Koordinatensystem. Die einzelnen Datenpunkte sind durch gerade Linien verbunden.} 
\item GUI - \textit{Eine grafische Benutzeroberfl\"ache (GBO oder GUI) ist eine Software-Komponente, die dem Benutzer eines Computers die Interaktion mit der Maschine \"uber grafische Symbole erlaubt.} 
\item $\theta$ - \textit{Nutation}
\item $\phi$ - \textit{Pr\"azession}
\item $\psi$ - \textit{Rotation}
\item $x_c$ - \textit{x-Koordinate}
\item $y_c$ - \textit{y-Koordintate}
\item $\dot \theta$ - \textit{Nutationsgeschwindigkeit}
\item $\dot \phi$ - \textit{Pr\"azessionsgeschwindigkeit}
\item $\dot \psi$ - \textit{Rotationsgeschwindigkeit}
\item $v_x$ - \textit{Geschwindigkeit in x-Richtung}
\item $v_y$ - \textit{Geschwindigkeit in y-Richtung}
\item $R$ - \textit{Radius}
\item $k$ - \textit{Reibungskoeffizient}
\item $a$ - \textit{Abstand vom Mittelpunkt zum Schwerpunkt}
\item $m$ - \textit{Masse des Kreisels}
\item $G$ - \textit{Erhaltungsgr\"o\ss e}
\item $atol$ - \textit{absolute Toleranz des Runge-Kutta-Verfahrens}
\item $rtol$ - \textit{relative Toleranz des Runge-Kutta-Verfahrens}
\end{itemize}



