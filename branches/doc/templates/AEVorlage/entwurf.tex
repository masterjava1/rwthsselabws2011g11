\chapter{Entwurf}
\label{ch:3}

\section{Grobentwurf: Subsysteme}
\label{sec:3.1}

{\em Grobe Strukturierung in Subsysteme (evtl. Namensbereiche), z.B.
Daten, Numerik, Grafik} 

\subsection{Statik}

\subsection{Dynamik}

\section{Detailentwurf: Klassen}
\label{sec:3.2}

{\em Klassenmodell (Statik: Klassendiagramm; Dynamik: Sequenzdiagramm, 
CRC Karten) inklusive (Klassen- bzw. Objekt-)Daten und 
(Klassen- bzw. Objekt-)Funktionen; Navigierbarkeit; Zugriffsrechte;
Abstrakte Datentypen:
vollst\"andig spezifizierte fundamentale Datentypen, z.B. {\tt vektor}, 
{\tt matrix}; Axiome, Bedingungen, Testfunktionen}
\\
\\
\begin{itemize}
\item Parser basierend auf {\tt flex}\footnote{\tt http://www.gnu.org/software/flex/manual/} und
{\tt bison}\footnote{\tt http://www.gnu.org/software/bison/manual/}
\cite{Aho1986CPT,Muchnick1997ACD} zur Eingabe der Daten (Problemspezifikation)
\item Newton's Methode zur L\"osung des nichtlinearen Systems \cite{Kelley2003SNE}
\item Direkte \cite{Golub1989MC} bzw. indirekte Verfahren \cite{Saad1986Agm}
zur L\"osung des linearen Systems
\item Automatisches Differenzieren \cite{Griewank2000EDP,Wengert1964ASA} zur 
Berechnung der Jacobimatrix bzw. des Produkts der Jacobimatrix mit einem Vektor
\end{itemize}

\subsection{Statik}

\subsection{Dynamik}


