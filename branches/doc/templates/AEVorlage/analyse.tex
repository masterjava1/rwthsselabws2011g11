\chapter{Analyse}
\label{ch:2}

\section{Anforderungsanalyse}
\label{sec:2.1}

\subsection{Benutzeranforderungen}

{\em Ausf\"uhrliche Beschreibung der Aufgabenstellung (Ziel: Konsens zwischen
Auftraggeber und Auftragnehmer)} \\
\\
L\"osung nichtlinearer Gleichungssysteme resultierend aus Anwendung
mit Newton's Methode \cite{Heath1998SCA,Kelley2003SNE};
Problemspezifikation; Weiterverarbeitung der Resultate

\subsection{Anwendungsfallanalyse}

{\em Anwendungsf\"alle (Statik: Anwendungsfalldiagramme; Dynamik: 
Aktivit\"atsdiagramme; Textuelle Beschreibungen laut Vorlage}

\subsubsection{Systemanforderungen}

{\em Beschreibung funktionaler und nichtfunktionaler Anforderungen basierend
auf Anwendungsf\"allen (Ziel: Konsens zwischen Auftraggeber und Auftragnehmer)}

\subsubsection{Benutzerdokumentation}

{\em kann basierend auf den Anwendungsf\"allen erstellt werden; siehe
\refchapter{ch:4}}

\section{Begriffsanalyse}

{\em Identifikation von Klassenkandidaten basierend auf Anforderungen;
Assoziationen zwischen Klassenkandidaten (Aggregation, Komposition, Vererbung)
und Kardinalit\"aten}


