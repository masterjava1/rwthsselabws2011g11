\chapter{Analyse}
\label{ch:2}

\section{Anforderungsanalyse}
\label{sec:2.1}

\subsection{Benutzeranforderungen}

{Das von Herrn Professor Gauger gestellte Simulationsproblem umfasst die Erstellung einer Software zur Simulation eines Stehaufkreisels.

Die Simulation soll dabei einmal f\"ur den Fall verschwindender Reibung sowie einmal für den Fall einer reibungsbehafteten Bewegung durchgef\"uhrt werden. 

Im Speziellen wird ein Runge-Kutta56-Verfahren mit adaptiver Schrittweitensteuerung unter Betrachtung einer Erhaltungsgröße (\textit{conserved quantity}) zur Simulation des Problems verwendet. Die Realisierung der Simulation findet in C++ statt.

Die Bedienung sowie die Auswertung der Ergebnisse der Simulationssoftware muss durch eine grafische Benutzeroberfl\"ache (\textit{GUI}) m\"oglich sein.

Die Simulationsergebnisse k\"onnen in einer \textit{ASCII}-formatierten Datei zur weiteren Verarbeitung und Auswertung exportiert werden.

%(Bilder exportieren?) 

Durch den modularen Aufbau ist die Wartbarkeit und sp\"atere Modifikationen oder Erweiterungen durch externe Mitarbeiter gew\"ahrleistet. 


Im weiteren umfasst der Arbeitsauftrag die Erstellung einer Benutzerdokumentation f\"ur den Endanwender. 
\em 

Das Hauptproblem besteht im L\"osen der Rechten Seite mithilfe des Runge-Kutta56-Verfahrens:

\begin{align*}
	&
	\ddot \theta (I+ma^2\sin^2\theta+kma\sin \theta(R-a\cos\theta)(-\dot x_c
	\sin\phi+\dot y_c \cos \phi-(R-a\cos\theta)\dot \theta))
	\\&
	=\underbrace{-(I_3-I)\dot \phi^2\sin\theta\cos\theta}_{=0}-I_3\dot \phi \sin \theta \dot \psi 
	+ (g+a\dot \theta^2\cos \theta)(-ma\sin\theta - km(R-a\cos \theta)
	\\&
	(-\dot x_c \sin \phi +\dot y_c \cos \phi-(R-a\cos\theta)\dot \theta))
\end{align*}
\begin{align*}
	&
	\ddot \phi I \sin\theta= -\underbrace{(2I-I_3)}_{=I}
	\dot \phi \dot \theta \cos \theta + I_3\dot \theta \dot \psi
	\\&
	-km(g+a\cos\theta\dot\theta^2+a\sin\theta\ddot\theta)
	(a-R\cos\theta)(\dot x_c\cos \phi +\dot y_c \sin\phi+
	(a\dot \phi+\dot \psi R)\sin \theta)
\end{align*}
\begin{align*}
	&\ddot \psi I_3=-I_3(\ddot \phi \cos \theta - \dot \phi \dot \theta \sin \theta)
	\\&
	-km(g+a\cos\theta\dot\theta^2+a \sin\theta\ddot \theta)(R\sin\theta)
	(\dot x_c\cos \phi +\dot y_c \sin \phi+(a\dot \phi +\dot \psi R)\sin\theta)
\end{align*}
\begin{align*}
	&m\ddot x_c=-km(g+a\cos\theta\dot \theta^2+a\sin \theta\ddot\theta)
	(\dot x_c+(a\dot \phi + \dot \psi R)\sin \theta\cos \phi+
	(a\cos \theta -R)\sin \phi \dot \theta)
\end{align*}
\begin{align*}
	&m\ddot y_c=-km(g+a\cos \theta\dot \theta^2+a \sin \theta \ddot \theta)
	(\dot y_c+(a\dot \phi + \dot \psi R)\sin \theta \cos \phi
	+(R-a\cos \theta)\cos\phi \dot \theta)
\end{align*} }

\subsection{Anwendungsfallanalyse}

{\em Anwendungsf\"alle (Statik: Anwendungsfalldiagramme; Dynamik: 
Aktivit\"atsdiagramme; Textuelle Beschreibungen laut Vorlage}

\subsubsection{Systemanforderungen}

{\em Beschreibung funktionaler und nichtfunktionaler Anforderungen basierend
auf Anwendungsf\"allen (Ziel: Konsens zwischen Auftraggeber und Auftragnehmer)}

\subsubsection{Benutzerdokumentation}

{\em kann basierend auf den Anwendungsf\"allen erstellt werden; siehe
\refchapter{ch:4}}

\section{Begriffsanalyse}

{\em Identifikation von Klassenkandidaten basierend auf Anforderungen;
Assoziationen zwischen Klassenkandidaten (Aggregation, Komposition, Vererbung)
und Kardinalit\"aten}


