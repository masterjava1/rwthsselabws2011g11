\chapter{Vorwort}
\label{ch:1}

\section{Aufgabenstellung und Struktur des Dokuments}
\label{sec:1.1}

\subsection*{Aufgabenstellung}

{Im Rahmen des Softwareentwicklungspraktikums (CES$\_$SS2012) soll eine Software zur Simulation eines Stehaufkreisels erstellt werden.
Die Simulationssoftware muss sowohl den reibungsfreien, als auch den reibungsbehafteten Fall korrekt simulieren k\"onnen.

Als Programmiersprache soll C++ verwendet werden. Der Quellcode soll derart strukturiert
und kommentiert sein, dass sp\"atere Modifikationen und Erweiterungen durch Dritte m\"oglich sind.}

\section{Projektmanagement}
\label{sec:1.2}
{
\begin{tabular}{l || l}
  \hline
  \hline
	Protoyping (MATLAB/ FORTRAN)& Alexander \\
	\hline
	Dokumentation &Lena \\
	\hline
	\hline
	Coding:& \\
	\hline
	Parameterset, Solver, Solution, Rkv56Parset, Rkv56,\\ DESolution,\textless  \textless interface\textgreater \textgreater RightSide, RHS, Rkv56Modified & Alexander\\
	\hline
	\textless \textless interface \textgreater \textgreater OutputInterface, OutputToolbox, Main, ExceptionHandlingModule,\\ MathException, NonCriticalME, CriticalME, ParameterException & William\\
	\hline
	GUI & Lena \\
	\hline
	\hline
\end{tabular}
}

%% \section{Lob und Kritik}
%% \label{sec:1.3}

%% {\em - folgt -} ?????????????????????

